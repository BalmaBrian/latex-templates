\documentclass[letterpaper,twocolumn,draft]{article}
% This is a LaTeX preamble file for ACM papers.
% This file should be used to prepare papers for any ACM.
% Please note this is a personal preamble file for ACM papers.
% The requirements for the format are given in this site: https://ai.stanford.edu/~suresh/theory/acmstyle.html#:~:text=Page%20limit%20is%2012%20pages,Here%20is%20the%20copyright%20form.
% They are as follows:
%   - 9pt font
%   - 10pt leading
%   - 0.33in column separation
%   - 3.33in textwidth
%   - 0.75in left, right, and top margins
%   - 1.00in bottom margin
%   - 1.00in bottom margin for the first column for ACM Copyright
%   - no page numbers
% \documentclass[letterpaper,twocolumn,draft]{article}
\usepackage{times}
\renewcommand{\normalsize}{\fontsize{9pt}{10pt}\selectfont}
\setlength{\columnsep}{0.33in}
\setlength{\textwidth}{3.33in}
\usepackage[margin=0.75in,top=0.75in,bottom=1.00in]{geometry}
\usepackage{titlesec}
\titleformat{\section}
  {\normalfont\Large\bfseries}{\thesection}{1em}{\vspace{-0.09in}}[\vspace{-0.09in}]
\titleformat{\subsection}
  {\normalfont\large\bfseries}{\thesubsection}{1em}{\vspace{-0.09in}}[\vspace{-0.09in}]
\titleformat{\subsubsection}
  {\normalfont\normalsize\bfseries}{\thesubsubsection}{1em}{\vspace{-0.09in}}[\vspace{-0.09in}]
\usepackage{textcase}
\usepackage{draftwatermark}
\SetWatermarkText{DRAFT}
\SetWatermarkScale{1}
\usepackage{url}
\usepackage{lipsum}

\begin{filecontents}{\jobname.bib}
@Book{test,
  author =       {Arpaci-Dusseau, Remzi H. and Arpaci-Dusseau Andrea C.},
  title =        {Operating Systems: Three Easy Pieces},
  publisher =    {Arpaci-Dusseau Books, LLC},
  year =         2015,
  edition =      {1.00},
  note =         {\url{http://pages.cs.wisc.edu/~remzi/OSTEP/}}}
\end{filecontents}

\begin{document}
% Title with author information section----------------------------------------
\title{
    Formatting a Generic ACM Paper:\\
    An (Incomplete) Example}
\date{}
\author{
    First Author\\
    First Author Institution\\
    City, State ZIP, Country\\
    email\\
    \and
    Second Author\\
    Second Author Institution\\
    City, State ZIP, Country\\
    email\\
    \and
    Third Author\\
    Third Author Institution\\
    City, State ZIP, Country\\
    email\\
    \and
    Fourth Author\\
    Fourth Author Institution\\
    City, State ZIP, Country\\
    email\\
    \and
    Fifth Author\\
    Fifth Author Institution\\
    City, State ZIP, Country\\
    email\\
    \and
    Sixth Author\\
    Sixth Author Institution\\
    City, State ZIP, Country\\
    email\\
    \and
    Seventh Author\\
    Seventh Author Institution\\
    City, State ZIP, Country\\
    email\\
    \and
    Eighth Author\\
    Eighth Author Institution\\
    City, State ZIP, Country\\
    email\\}
\maketitle

% Abstract section-------------------------------------------------------------
\section*{\MakeUppercase{Abstract}}
Your abstract text goes here. Just a few facts. Whet our appetites.
Not more than 200 words, if possible, and preferably closer to 150.
\lipsum[1-2]

% CCS Concepts-----------------------------------------------------------------
\section*{\textbf{CCS Concepts}}
\textbf{
    $\cdot$ Security and privacy $\rightarrow$ \textit{General and reference};
    $\cdot$ Human-centered computing $\rightarrow$ \textit{Collaborative and 
    social computing systems and tools}}

% Keywords---------------------------------------------------------------------
\section*{\textbf{Keywords}}
Keywords go here; separated by semicolons; end with a period.

% Introduction section---------------------------------------------------------
\section{\MakeUppercase{Introduction}}
A paragraph of text goes here. Lots of text. Plenty of interesting
text. Text text text text text text text text text text text text text
text text text text text text text text text text text text text text
text text text text text text text text text text text text text text
text text text text text text text.
More fascinating text. Features galore, plethora of promises.

% Info on CCS & Keywords-------------------------------------------------------
\section{\MakeUppercase{What are the CCS Concepts \& Keywords?}}
The ACM Computing Classification System (CCS) is a system that allows authors 
to classify their work. The CCS is hierarchical, and authors are asked to 
select the CCS concepts that are most relevant to their work. A full list 
of CCS concepts and their descriptions can be found at 
\url{https://dl.acm.org/ccs/ccs_flat.cfm}.


Keywords are a list of terms that are relevant to your work. They are used 
for indexing and searching for your work. A good set of keywords will help 
others find your work. Keywords should be separated by semicolons and end 
with a period.

% Sections---------------------------------------------------------------------
\section{\MakeUppercase{Sections}}
Sections are numbered, and the numbering is displayed. Top level sections must 
be in all caps. Subsections are also numbered and displayed. Subsubsections 
are also numbered and displayed. Subsections \& subsubsections don't need to
be in all caps.

% Footnotes, Verbatim, and Citations section-----------------------------------
\section{\MakeUppercase{Footnotes, Verbatim, and Citations}}
Footnotes should be places after puctuation characters, without any space 
between side characters and footnotes, like so.
\footnote{
    Remember that USENIX format stopped using endnotes and is not using regular
    footnotes.}
Ans some embedded literal code may look as follows:
\begin{verbatim}
    #include <stdio.h>
    int main() {
        printf("Hello, World!\n");
        return 0;
    }
\end{verbatim}

Now we're going to cite some somebody. Watch for the cite tag. Here it comes.
Arpachi-Desseau and Arpachi-Dusseau coauthored an excellent OS book, which is 
also really funny~\cite{test}, and Waldsurger got into the 
SIGOPS hall-of-fame due to his seminal paper about resource management in the
ESX hypervisor~\cite{test}.

% Bibliography-----------------------------------------------------------------
\bibliographystyle{plain}
\bibliography{\jobname}
\end{document}